\hypertarget{index_intro_sec}{}\section{Introduction}\label{index_intro_sec}
The \href{http://octomap.github.com/}{\tt Octo\+Map library} implements a 3\+D occupancy grid mapping approach. It provides data structures and mapping algorithms. The map is implemented using an Octree. It is designed to meet the following requirements\+: 


\begin{DoxyItemize}
\item {\bfseries Full 3\+D model.} The map is able to model arbitrary environments without prior assumptions about it. The representation models occupied areas as well as free space. If no information is available about an area (commonly denoted as {\itshape unknown areas}), this information is encoded as well. While the distinction between free and occupied space is essential for safe robot navigation, information about unknown areas is important, e.\+g., for autonomous exploration of an environment.  
\item {\bfseries Updatable.} It is possible to add new information or sensor readings at any time. Modeling and updating is done in a {\itshape probabilistic} fashion. This accounts for sensor noise or measurements which result from dynamic changes in the environment, e.\+g., because of dynamic objects. Furthermore, multiple robots are able to contribute to the same map and a previously recorded map is extendable when new areas are explored.


\item {\bfseries Flexible.} The extent of the map does not have to be known in advance. Instead, the map is dynamically expanded as needed. The map is multi-\/resolution so that, for instance, a high-\/level planner is able to use a coarse map, while a local planner may operate using a fine resolution. This also allows for efficient visualizations which scale from coarse overviews to detailed close-\/up views.  
\item {\bfseries Compact.} The is stored efficiently, both in memory and on disk. It is possible to generate compressed files for later usage or convenient exchange between robots even under bandwidth constraints.  
\end{DoxyItemize}

Octomap was developed by \href{http://www.informatik.uni-freiburg.de/~wurm}{\tt Kai M. Wurm} and \href{http://www.informatik.uni-freiburg.de/~hornunga}{\tt Armin Hornung}, and is currently maintained by Armin Hornung. A tracker for bug reports and feature requests is available available \href{https://github.com/OctoMap/octomap/issues}{\tt on Git\+Hub}. You can find an overview at \href{http://octomap.github.com/}{\tt http\+://octomap.\+github.\+com/} and the code repository at \href{https://github.com/OctoMap/octomap}{\tt https\+://github.\+com/\+Octo\+Map/octomap}.\hypertarget{index_install_sec}{}\section{Installation}\label{index_install_sec}
See the file R\+E\+A\+D\+M\+E.\+txt in the main folder. \hypertarget{index_changelog_sec}{}\section{Changelog}\label{index_changelog_sec}
See the file C\+H\+A\+N\+G\+E\+L\+O\+G.\+txt in the main folder or the \href{https://raw.github.com/OctoMap/octomap/master/octomap/CHANGELOG.txt}{\tt latest version online}. \hypertarget{index_gettingstarted_sec}{}\section{Getting Started}\label{index_gettingstarted_sec}
Jump right in and have a look at the main class octomap\+::\+Oc\+Tree Oc\+Tree and the examples in src/octomap/simple\+\_\+example.\+cpp. To integrate single measurements into the 3\+D map have a look at Oc\+Tree\+:\+:insert\+Ray(...), to insert full 3\+D scans (pointclouds) please have a look at Oc\+Tree\+:\+:insert\+Point\+Cloud(...). Queries can be performed e.\+g. with Oc\+Tree\+:\+:search(...) or Oc\+Tree\+:\+:cast\+Ray(...). The preferred way to batch-\/access or process nodes in an Octree is with the iterators leaf\+\_\+iterator, tree\+\_\+iterator, or leaf\+\_\+bbx\+\_\+iterator.



The Oc\+Tree class is derived from Occupancy\+Oc\+Tree\+Base, with most functionality in the parent class. Also derive from Occupancy\+Oc\+Tree\+Base if you you want to implement your own Octree and node classes. You can have a look at the classes Oc\+Tree\+Stamped and Oc\+Tree\+Node\+Stamped as examples. 

Start the 3\+D visualization with\+: {\bfseries bin/octovis} 

You will find an example 3\+D scan (please bunzip2 first) and an example Octo\+Map .bt file in the directory {\bfseries share/data} to try. More data sets are available at \href{http://ais.informatik.uni-freiburg.de/projects/datasets/octomap/}{\tt http\+://ais.\+informatik.\+uni-\/freiburg.\+de/projects/datasets/octomap/}. 